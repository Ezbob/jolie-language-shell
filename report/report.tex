\documentclass[12pt]{article}
\usepackage[english]{babel}
\usepackage[utf8]{inputenc}
\usepackage{graphicx}
\usepackage{listings}
\usepackage{fancyhdr}
\usepackage[a4paper,top=2.1cm,bottom=2.1cm,left=2.4cm,right=2.4cm]{geometry}

\title{ \textbf{DM848}: Microservice Programming \\ {\Large Sandboxed Interpreter for Interactive Code Execution} }
\author{Anders Busch, anbus12@student.sdu.dk}
\date{\today \\ \hrulefill{}}

%New command for inserting images put the images in pics subfolder
%synopsis: \img{ scale }{ image name }{ caption text }
\newcommand{\img}[3] {
	\begin{center}
	\includegraphics[scale=#1]{pics/#2}\\
	{\small #3}
	\end{center}
}

%New command for inserting pdfs put the pdfs in pics subfolder
%synopsis: \pdf{ scale }{ page number }{ image name }{ caption text }
\newcommand{\pdf}[4] {
	\begin{center}
	\includegraphics[scale=#1,page=#2]{pics/#3}\\
	{\small #4}
	\end{center}
}

% code listing options, can be "overload" by cascading 
\lstset{breaklines=true, tabsize=4, numbers=left, title=\lstname, language=c}

\newcommand{\docker}[0] {\textsc{docker}}
\newcommand{\jolie}[0] {\textsc{jolie}}

\renewcommand{\baselinestretch}{1.5}

\begin{document}

\maketitle

\section{Introduction}

When learning a new programming language, it is often useful to have a environment where the learner can execute code, quick and easy without having to worry about breaking the execution environment or installing new software locally. \\

\noindent{}In this report we will describe the design and implementation of a cloud server, for executing \jolie{} code, embedded using the popular virtualization application \docker{}. 

\section{Preliminaries}

\section{Technical Description}

\section{Related Work and Discussion}

\end{document}
